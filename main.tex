\documentclass[polish, 11pt]{article}

\usepackage[a4paper,
  top=.5cm,
  bottom=2.5cm,
  left=2.5cm,
  right=2.5cm,
  headheight=86pt, % as per the warning by fancyhdr
  includehead,includefoot,
  heightrounded, % to avoid spurious underfull messages
]{geometry}
\usepackage{babel}
\usepackage{polski}
\usepackage[utf8]{inputenc}
\usepackage[T1]{fontenc}
\usepackage{graphicx}
\usepackage{float}
\usepackage{minted} % na razie zostawiłem ale do pseudokodu minted chyba nie ma wsparcia - jest coś jak algorithm2e ale nie przyglądałem sie bardziej albo po prostu listings
\usemintedstyle{xcode}
\usepackage{booktabs}
\usepackage{multirow}

\usepackage{fancyhdr}
\pagestyle{fancy}
\fancyhf{}
\fancyheadoffset[L]{2cm}
\lhead{\includegraphics[height=2cm]{figures/PWRlogo.jpg}\\
    \vspace{3mm}
    Tytuł projektu: Apapp
    \vspace{1mm}
}
\cfoot{Wrocław, 2019}
\rfoot{\thepage}

\begin{document}
{
    \centering
    \Huge{PROJEKTOWANIE SYSTEMÓW INTERNETOWYCH I MOBILNYCH}
\vspace{2cm}

    \huge{WYDZIAŁ ELEKTRONIKI - INFORMATYKA \\ Systemy i Sieci Komputerowe}
\vspace{2cm}

    \LARGE{\textbf{Tytuł projektu: Apapp}}
\vspace{3cm}

    \begin{flushright}
        Zespół projektowy:\\
        Janusz Długosz, 235746\\
        Jakub Dorda, 235013\\
        Marcin Kotas, 235098\\
        Mateusz Polok, 235???
        
    \end{flushright}
}
\newpage
\tableofcontents
\newpage

\section{Cel i zakres projektu}

\section{Technologie informatyczne wykorzystane w projekcie}

\section{Harmonogram prac projektowych – wykres Gantta}

\section{Architektura aplikacji}

\section{Opis struktury kodowania aplikacji}

\section{Prezentacja aplikacji}

\section{Analiza SWOT powdrożeniowa}

\end{document}